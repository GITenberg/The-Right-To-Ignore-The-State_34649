% %%%%%%%%%%%%%%%%%%%%%%%%%%%%%%%%%%%%%%%%%%%%%%%%%%%%%%%%%%%%%%%%%%%%%%% %
%                                                                         %
% Project Gutenberg's The Right To Ignore The State, by Herbert Spencer   %
%                                                                         %
% This eBook is for the use of anyone anywhere at no cost and with        %
% almost no restrictions whatsoever.  You may copy it, give it away or    %
% re-use it under the terms of the Project Gutenberg License included     %
% with this eBook or online at www.gutenberg.org                          %
%                                                                         %
%                                                                         %
% Title: The Right To Ignore The State                                    %
%                                                                         %
% Author: Herbert Spencer                                                 %
%                                                                         %
% Release Date: December 14, 2010 [EBook #34649]                          %
%                                                                         %
% Language: English                                                       %
%                                                                         %
% Character set encoding: ASCII                                           %
%                                                                         %
% *** START OF THIS PROJECT GUTENBERG EBOOK THE RIGHT TO IGNORE THE STATE ***
%                                                                         %
% %%%%%%%%%%%%%%%%%%%%%%%%%%%%%%%%%%%%%%%%%%%%%%%%%%%%%%%%%%%%%%%%%%%%%%% %

\def\ebook{34649}
%%%%%%%%%%%%%%%%%%%%%%%%%%%%%%%%%%%%%%%%%%%%%%%%%%%%%%%%%%%%%%%%%%%%%%%%%%%
%% The Right to Ignore the State. By Herbert Spencer.                    %%
%%                                                                       %%
%% Packages and substitutions:                                           %%
%%                                                                       %%
%% amsmath     Basic AMS math package.                                   %%
%% book        Document class.                                           %%
%% inputenc    Encoding                                                  %%
%% verbatim    Preformated text                                          %%
%%                                                                       %%
%% PDF Pages: 25                                                         %%
%%                                                                       %%
%% Compile sequence:                                                     %%
%%         pdflatex                                                      %%
%%                                                                       %%
%% Compile History:                                                      %%
%%                                                                       %%
%% Dec 10: Laverock. Compiled with pdflatex:                             %%
%%         [pdfeTeX, Version 3.1415926-1.40.9 (MiKTeX 2.7)]              %%
%%                                                                       %%
%%                                                                       %%
%% December 2010: pglatex.                                               %%
%%   Compile this project with:                                          %%
%%   pdflatex 34649-t.tex                                                %%
%%                                                                       %%
%%   pdfTeXk, Version 3.141592-1.40.3 (Web2C 7.5.6)                      %%
%%                                                                       %%
%%%%%%%%%%%%%%%%%%%%%%%%%%%%%%%%%%%%%%%%%%%%%%%%%%%%%%%%%%%%%%%%%%%%%%%%%%%

\documentclass[oneside]{book}
\usepackage[latin1]{inputenc}[2006/05/05]
\usepackage{amsmath}[2000/07/18]
\usepackage{verbatim}[2003/08/22]             % preformated text
\listfiles

\newcommand{\DoubleLine}{%
\medskip
\begin{center}\rule{1in}{0.5pt}\end{center}
\vspace{-6ex}
\begin{center}\rule{1in}{0.5pt}\end{center}
\medskip
}

\newcommand{\SingleLine}{%
\medskip
\begin{center}\rule{1in}{0.5pt}\end{center}
\medskip
}

% For sensible insertion of boilerplate/licence
% overlong lines will wrap and be indented 0.25in
% and text is set in "small" size
\makeatletter
\def\@makeschapterhead#1{%
  \vspace*{10\p@}%
  {\parindent \z@ \centering
    \normalfont
    \interlinepenalty\@M
    \huge \bfseries  #1\par\nobreak
    \vskip 20\p@
  }}
\renewcommand*\l@section{\@dottedtocline{1}{0pt}{2.3em}}
\renewcommand\@pnumwidth{2.55em}
\def\@xobeysp{~\hfil\discretionary{}{\kern\z@}{}\hfilneg}
\renewcommand\verbatim@processline{\leavevmode
  \null\kern-0.25in\the\verbatim@line\par}
\addto@hook\every@verbatim{\@totalleftmargin0.25in\small}

\pagestyle{empty}
\begin{document}
\begin{verbatim}
Project Gutenberg's The Right To Ignore The State, by Herbert Spencer

This eBook is for the use of anyone anywhere at no cost and with
almost no restrictions whatsoever.  You may copy it, give it away or
re-use it under the terms of the Project Gutenberg License included
with this eBook or online at www.gutenberg.org


Title: The Right To Ignore The State

Author: Herbert Spencer

Release Date: December 14, 2010 [EBook #34649]

Language: English

Character set encoding: ASCII

*** START OF THIS PROJECT GUTENBERG EBOOK THE RIGHT TO IGNORE THE STATE ***


Produced by Fritz Ohrenschall, Keith Edkins and the Online
Distributed Proofreading Team at http://www.pgdp.net (This
file was produced from images generously made available
by The Internet Archive)
\end{verbatim}
\newpage
%-----File: 001.png



%<|select,none>
%/*
%Spencer, Herbert


%The right to ignore the
%state
%*/

%<|select,all>

%-----File: 002.png
%-----File: 003.png




\begin{center}{\large Freedom Pamphlet.}\end{center}

\hrule

\vspace{\baselineskip}

\begin{center}{\large PRICE ONE PENNY.}\end{center}

\vspace{\baselineskip}

\begin{center}{\Huge THE RIGHT TO IGNORE}\end{center}
\begin{center}{\Huge THE STATE.}\end{center}

\vspace{\baselineskip}

\begin{center}{BY}\end{center}

\begin{center}{\large HERBERT SPENCER.}\end{center}

\vspace{\baselineskip}

\begin{center}\rule{2in}{0.5pt}\end{center}

\vspace{\baselineskip}

\begin{center}{(\textit{Reprinted from ``Social Statics,'' 1850 Edition.})}\end{center}

\vspace{\baselineskip}

\hrule

\begin{center}{\textsc{London.}}\end{center}

\begin{center}{\textsc{Freedom Press, 127 Ossulston Street, N. W.}}\end{center}

\begin{center}{1913.}\end{center}


\newpage
%-----File: 004.png


[It is only fair to the memory of Mr.\ Herbert Spencer that
we should warn the reader of the following chapter from the
original edition of Mr.\ Spencer's ``Social Statics,'' written in
1850, that it was omitted by the author from the revised edition,
published in 1892. We may legitimately infer that this omission
indicates a change of view. But to repudiate is not to answer,
and Mr.\ Spencer never answered his arguments for the right to
ignore the State. It is the belief of the Anarchists that these
arguments are unanswerable.]


\newpage\pagestyle{plain}

%-----File: 005.png




\begin{center}{\LARGE The Right to Ignore the State.}\end{center}


\S{} 1. As a corollary to the proposition that all institutions
must be subordinated to the law of equal freedom, we cannot
choose but admit the right of the citizen to adopt a condition of
voluntary outlawry. If every man has freedom to do all that
he wills, provided he infringes not the equal freedom of any
other man, then he is free to drop connection with the State,---to
relinquish its protection and to refuse paying towards its
support. It is self-evident that in so behaving he in no way
trenches upon the liberty of others; for his position is a passive
one, and, whilst passive, he cannot become an aggressor. It is
equally self-evident that he cannot be compelled to continue one
of a political corporation without a breach of the moral law,
seeing that citizenship involves payment of taxes; and the
taking away of a man's property against his will is an infringement
of his rights. Government being simply an agent employed
in common by a number of individuals to secure to them certain
advantages, the very nature of the connection implies that it is
for each to say whether he will employ such an agent or not.
If any one of them determines to ignore this mutual-safety
confederation, nothing can be said, except that he loses all claim
to its good offices, and exposes himself to the danger of maltreatment,---a
thing he is quite at liberty to do if he likes. He
cannot be coerced into political combination without a breach of
the law of equal freedom; he \textit{can} withdraw from it without
committing any such breach; and he has therefore a right so to
withdraw.


\S{} 2. ``No human laws are of any validity if contrary to
the law of nature: and such of them as are valid derive all their
force and all their authority mediately or immediately from this
\newpage\noindent
%-----File: 006.png
original.'' Thus writes Blackstone, to whom let all honour be
given for having so far outseen the ideas of his time,--and,
indeed, we may say of our time. A good antidote, this, for those
political superstitions which so widely prevail. A good check
upon that sentiment of power-worship which still misleads us by
magnifying the prerogatives of constitutional governments as it
once did those of monarchs. Let men learn that a legislature is
\textit{not} ``our God upon earth,'' though, by the authority they ascribe
to it and the things they expect from it, they would seem to
think it is. Let them learn rather that it is an institution
serving a purely temporary purpose, whose power, when not
stolen, is, at the best, borrowed.

Nay, indeed, have we not seen that government is essentially
immoral? Is it not the offspring of evil, bearing about it all
the marks of its parentage? Does it not exist because crime
exists? Is it not strong, or, as we say, despotic, when crime is
great? Is there not more liberty---that is, less government---as
crime diminishes? And must not government cease when crime
ceases, for very lack of objects on which to perform its function?
Not only does magisterial power exist \textit{because} of evil, but it
exists \textit{by} evil. Violence is employed to maintain it; and all
violence involves criminality. Soldiers, policemen, and gaolers;
swords, batons, and fetters,---are instruments for inflicting pain;
and all infliction of pain is, in the abstract, wrong. The State
employs evil weapons to subjugate evil, and is alike contaminated
by the objects with which it deals and the means by which it
works. Morality cannot recognise it; for morality, being simply
a statement of the perfect law, can give no countenance to anything
growing out of, and living by, breaches of that law.
Wherefore legislative authority can never be ethical---must
always be conventional merely.

Hence there is a certain inconsistency in the attempt to
determine the right position, structure, and conduct of a government
by appeal to the first principles of rectitude. For, as just
pointed out, the acts of an institution which is, in both nature
and origin, imperfect cannot be made to square with the perfect
law. All that we can do is to ascertain, firstly, in what attitude
\newpage\noindent
%-----File: 007.png
a legislature must stand to the community to avoid being by its
mere existence an embodied wrong; secondly, in what manner it
must be constituted so as to exhibit the least incongruity with
the moral law; and, thirdly, to what sphere its actions must be
limited to prevent it from multiplying those breaches of equity it
is set up to prevent.

The first condition to be conformed to before a legislature
can be established without violating the law of equal freedom is
the acknowledgment of the right now under discussion---the
right to ignore the State.


\S{} 3. Upholders of pure despotism may fitly believe State-control
to be unlimited and unconditional. They who assert
that men are made for governments and not governments for
men may consistently hold that no one can remove himself
beyond the pale of political organisation. But they who
maintain that the people are the only legitimate source of power---that
legislative authority is not original, but deputed---cannot
deny the right to ignore the State without entangling themselves
in an absurdity.

For, if legislative authority is deputed, it follows that those
from whom it proceeds are the masters of those on whom it is
conferred: it follows further that as masters they confer the said
authority voluntarily: and this implies that they may give or
withhold it as they please. To call that deputed which is
wrenched from men whether they will or not is nonsense. But
what is here true of all collectively is equally true of each
separately. As a government can rightly act for the people only
when empowered by them, so also can it rightly act for the
individual only when empowered by him. If A, B, and C
debate whether they shall employ an agent to perform for them
a certain service, and if, whilst A and B agree to do so, C
dissents, C cannot equitably be made a party to the agreement in
spite of himself. And this must be equally true of thirty as of
three: and, if of thirty, why not of three hundred, or three
thousand, or three millions?
\newpage
%-----File: 008.png


\S{} 4. Of the political superstitions lately alluded to, none
is so universally diffused as the notion that majorities are
omnipotent. Under the impression that the preservation of
order will ever require power to be wielded by some party, the
moral sense of our time feels that such power cannot rightly be
conferred on any but the largest moiety of society. It interprets
literally the saying that ``the voice of the people is the voice of
God,'' and, transferring to the one the sacredness attached to the
other, it concludes that from the will of the people---that is, of
the majority---there can be no appeal. Yet is this belief entirely
erroneous.

\vspace{\baselineskip}

Suppose, for the sake of argument, that, struck by some
Malthusian panic, a legislature duly representing public opinion
were to enact that all children born during the next ten years
should be drowned. Does any one think such an enactment
would be warrantable? If not, there is evidently a limit to the
power of a majority. Suppose, again, that of two races living
together---Celts and Saxons, for example---the most numerous
determined to make the others their slaves. Would the authority
of the greatest number be in such case valid? If not, there is
something to which its authority must be subordinate. Suppose,
once more, that all men having incomes under \pounds{}50 a year were
to resolve upon reducing every income above that amount to
their own standard, and appropriating the excess for public
purposes. Could their resolution be justified? If not, it must
be a third time confessed that there is a law to which the
popular voice must defer. What, then, is that law, if not the
law of pure equity---the law of equal freedom? These restraints,
which all would put to the will of the majority, are exactly the
restraints set up by that law. We deny the right of a majority
to murder, to enslave, or to rob, simply because murder,
enslaving, and robbery are violations of that law---violations
too gross to be overlooked. But, if great violations of it are
wrong, so also are smaller ones. If the will of the many cannot
supersede the first principle of morality in these cases, neither
can it in any. So that, however insignificant the minority, and
\newpage\noindent
%-----File: 009.png
however trifling the proposed trespass against their rights, no
such trespass is permissible.

When we have made our constitution purely democratic,
thinks to himself the earnest reformer, we shall have brought
government into harmony with absolute justice. Such a faith,
though perhaps needful for the age, is a very erroneous one. By
no process can coercion be made equitable. The freest form of
government is only the least objectionable form. The rule of the
many by the few we call tyranny: the rule of the few by the
many is tyranny also, only of a less intense kind. ``You shall
do as we will, and not as you will,'' is in either case the declaration;
and, if the hundred make it to ninety-nine, instead of the
ninety-nine to the hundred, it is only a fraction less immoral.
Of two such parties, whichever fulfils this declaration necessarily
breaks the law of equal freedom: the only difference being that
by the one it is broken in the persons of ninety-nine, whilst by
the other it is broken in the persons of a hundred. And the
merit of the democratic form of government consists solely in
this,---that it trespasses against the smallest number.

The very existence of majorities and minorities is indicative
of an immoral state. The man whose character harmonises with
the moral law, we found to be one who can obtain complete
happiness without diminishing the happiness of his fellows. But
the enactment of public arrangements by vote implies a society
consisting of men otherwise constituted---implies that the desires
of some cannot be satisfied without sacrificing the desires of
others---implies that in the pursuit of their happiness the
majority inflict a certain amount of \textit{un}happiness on the minority---implies,
therefore, organic immorality. Thus, from another
point of view, we again perceive that even in its most equitable
form it is impossible for government to dissociate itself from
evil; and further, that, unless the right to ignore the State is
recognised, its acts must be essentially criminal.


\S{} 5. That a man is free to abandon the benefits and throw
off the burdens of citizenship, may indeed be inferred from the
admissions of existing authorities and of current opinion.
\newpage\noindent
%-----File: 010.png
Unprepared as they probably are for so extreme a doctrine as the
one here maintained, the Radicals of our day yet unwittingly
profess their belief in a maxim which obviously embodies this
doctrine. Do we not continually hear them quote Blackstone's
assertion that ``no subject of England can be constrained to
pay any aids or taxes even for the defence of the realm or the
support of government, but such as are imposed by his own
consent, or that of his representative in Parliament''? And
what does this mean? It means, say they, that every man
should have a vote. True: but it means much more. If there
is any sense in words, it is a distinct enunciation of the very
right now contended for. In affirming that a man may not be
taxed unless he has directly or indirectly given his consent, it
affirms that he may refuse to be so taxed; and to refuse to be
taxed is to cut all connection with the State. Perhaps it will
be said that this consent is not a specific, but a general, one,
and that the citizen is understood to have assented to every
thing his representative may do, when he voted for him. But
suppose he did not vote for him; and on the contrary did all
in his power to get elected some one holding opposite views---what
then? The reply will probably be that by taking part in
such an election, he tacitly agreed to abide by the decision of
the majority. And how if he did not vote at all? Why then
he cannot justly complain of any tax, seeing that he made no
protest against its imposition. So, curiously enough, it seems
that he gave his consent in whatever way he acted---whether
he said ``Yes,'' whether he said ``No,'' or whether he remained
neuter! A rather awkward doctrine, this. Here stands an
unfortunate citizen who is asked if he will pay money for a
certain proffered advantage; and, whether he employs the only
means of expressing his refusal or does not employ it, we are
told that he practically agrees, if only the number of others
who agree is greater than the number of those who dissent.
And thus we are introduced to the novel principle that A's
consent to a thing is not determined by what A says, but by
what B may happen to say!

It is for those who quote Blackstone to choose between this
\newpage\noindent
%-----File: 011.png
absurdity and the doctrine above set forth. Either his maxim
implies the right to ignore the State, or it is sheer nonsense.


\S{} 6. There is a strange heterogeneity in our political
faiths. Systems that have had their day, and are beginning
here and there to let the daylight through, are patched with
modern notions utterly unlike in quality and colour; and men
gravely display these systems, wear them, and walk about in
them, quite unconscious of their grotesqueness. This transition
state of ours, partaking as it does equally of the past and the
future, breeds hybrid theories exhibiting the oddest union of
bygone despotism and coming freedom. Here are types of the
old organisation curiously disguised by germs of the new---peculiarities
showing adaptation to a preceding state modified
by rudiments that prophesy of something to come---making
altogether so chaotic a mixture of relationships that there is no
saying to what class these births of the age should be referred.

As ideas must of necessity bear the stamp of the time, it
is useless to lament the contentment with which these incongruous
beliefs are held. Otherwise it would seem unfortunate
that men do not pursue to the end the trains of reasoning
which have led to these partial modifications. In the present
case, for example, consistency would force them to admit that, on
other points besides the one just noticed, they hold opinions and
use arguments in which the right to ignore the State is involved.

For what is the meaning of Dissent? The time was when
a man's faith and his mode of worship were as much determinable
by law as his secular acts; and, according to provisions
extant in our statute-book, are so still. Thanks to the growth
of a Protestant spirit, however, we have ignored the State in
this matter---wholly in theory, and partly in practice. But how
have we done so? By assuming an attitude which, if consistently
maintained, implies a right to ignore the State entirely.
Observe the positions of the two parties. ``This is your creed,''
says the legislator; ``you must believe and openly profess what
is here set down for you.'' ``I shall not do anything of the
kind,'' answers the Nonconformist; ``I will go to prison rather.''
\newpage\noindent
%-----File: 012.png
``Your religious ordinances,'' pursues the legislator, ``shall be
such as we have prescribed. You shall attend the churches
we have endowed, and adopt the ceremonies used in them.''
``Nothing shall induce me to do so,'' is the reply; ``I altogether
deny your power to dictate to me in such matters, and mean
to resist to the uttermost.'' ``Lastly,'' adds the legislator, ``we
shall require you to pay such sums of money toward the support
of these religious institutions as we may see fit to ask.'' ``Not
a farthing will you have from me,'' exclaims our sturdy Independent;
``even did I believe in the doctrines of your church
(which I do not), I should still rebel against your interference;
and, if you take my property, it shall be by force and under
protest.''

What now does this proceeding amount to when regarded
in the abstract? It amounts to an assertion by the individual
of the right to exercise one of his faculties---the religious
sentiment---without let or hindrance, and with no limit save
that set up by the equal claims of others. And what is meant
by ignoring the State? Simply an assertion of the right similarly
to exercise \textit{all} the faculties. The one is just an expansion
of the other---rests on the same footing with the other---must
stand or fall with the other. Men do indeed speak of civil
and religious liberty as different things: but the distinction
is quite arbitrary. They are parts of the same whole, and
cannot philosophically be separated.

``Yes they can,'' interposes an objector; ``assertion of the
one is imperative as being a religious duty. The liberty to
worship God in the way that seems to him right, is a liberty
without which a man cannot fulfil what he believes to be
divine commands, and therefore conscience requires him to
maintain it.'' True enough; but how if the same can be
asserted of all other liberty? How if maintenance of this also
turns out to be a matter of conscience? Have we not seen
that human happiness is the divine will---that only by exercising
our faculties is this happiness obtainable---and that it is impossible
to exercise them without freedom? And, if this freedom
for the exercise of faculties is a condition without which the
\newpage\noindent
%-----File: 013.png
divine will cannot be fulfilled, the preservation of it is, by our
objector's own showing, a duty. Or, in other words, it appears
not only that the maintenance of liberty of action \textit{may} be a
point of conscience, but that it \textit{ought} to be one. And thus we
are clearly shown that the claims to ignore the State in religious
and in secular matters are in essence identical.

The other reason commonly assigned for nonconformity
admits of similar treatment. Besides resisting State dictation
in the abstract, the Dissenter resists it from disapprobation of
the doctrines taught. No legislative injunction will make him
adopt what he considers an erroneous belief; and, bearing in
mind his duty toward his fellow-men, he refuses to help through
the medium of his purse in disseminating this erroneous belief.
The position is perfectly intelligible. But it is one which
either commits its adherents to civil nonconformity also, or
leaves them in a dilemma. For why do they refuse to be
instrumental in spreading error? Because error is adverse to
human happiness. And on what ground is any piece of secular
legislation disapproved? For the same reason---because thought
adverse to human happiness. How then can it be shown that
the State ought to be resisted in the one case and not in the
other? Will any one deliberately assert that, if a government
demands money from us to aid in \textit{teaching} what we think will
produce evil, we ought to refuse it, but that, if the money is for
the purpose of \textit{doing} what we think will produce evil, we ought
not to refuse it? Yet such is the hopeful proposition which
those have to maintain who recognise the right to ignore the
State in religious matters, but deny it in civil matters.


\S{} 7. The substance of this chapter once more reminds us
of the incongruity between a perfect law and an imperfect State.
The practicability of the principle here laid down varies directly
as social morality. In a thoroughly vicious community its
admission would be productive of anarchy.\footnote{Mr.\
Spencer here uses the word ``anarchy'' in the sense of disorder.}
In a completely
virtuous one its admission will be both innocuous and inevitable.
Progress toward a condition of social health---a condition, that
\newpage\noindent
%-----File: 014.png
is, in which the remedial measures of legislation will no longer
be needed---is progress toward a condition in which those
remedial measures will be cast aside, and the authority prescribing
them disregarded. The two changes are of necessity
co-ordinate. That moral sense whose supremacy will make
society harmonious and government unnecessary is the same
moral sense which will then make each man assert his freedom
even to the extent of ignoring the State---is the same moral
sense which, by deterring the majority from coercing the
minority, will eventually render government impossible. And,
as what are merely different manifestations of the same sentiment
must bear a constant ratio to each other, the tendency
to repudiate governments will increase only at the same rate
that governments become needless.

Let not any be alarmed, therefore, at the promulgation of
the foregoing doctrine. There are many changes yet to be
passed through before it can begin to exercise much influence.
Probably a long time will elapse before the right to ignore
the State will be generally admitted, even in theory. It will
be still longer before it receives legislative recognition. And
even then there will be plenty of checks upon the premature
exercise of it. A sharp experience will sufficiently instruct
those who may too soon abandon legal protection. Whilst, in
the majority of men, there is such a love of tried arrangements,
and so great a dread of experiments, that they will probably
not act upon this right until long after it is safe to do so.

\newpage
%-----File: 015.png




\begin{center}{\LARGE Anarchist Communism.\footnotemark}\end{center}

\begin{center}{ITS AIMS AND PRINCIPLES.}\end{center}

\footnotetext{It would be only fair to state that the Individualist school of
Anarchism, which includes many eminent writers and thinkers, differs
from us mainly on the question of Communism---\textit{i.e.}, on the holding of
property, the remuneration of labour, etc. Anarchism, however, affords
the opportunity for experiment in all these matters, and in that sense
there is no dispute between us.}

Anarchism may be briefly defined as the negation of all
government and all authority of man over man; Communism
as the recognition of the just claim of each to the fullest satisfaction
of all his needs---physical, moral, and intellectual. The
Anarchist, therefore, whilst resisting as far as possible all forms
of coercion and authority, repudiates just as firmly even the
suggestion that he should impose himself upon others, realising
as he does that this fatal propensity in the majority of mankind
has been the cause of nearly all the misery and bloodshed in the
world. He understands just as clearly that to satisfy his needs
without contributing, to the best of his ability, his share of
labour in maintaining the general well-being, would be to live at
the expense of others---to become an exploiter and live as the
rich drones live to-day. Obviously, then, government on the one
hand and private ownership of the means of production on the
other, complete the vicious circle---the present social system---which
keeps mankind degraded and enslaved.

There will be no need to justify the Anarchist's attack upon
\textit{all} forms of government: history teaches the lesson he has
learned on every page. But that lesson being concealed from
the mass of the people by interested advocates of ``law and
order,'' and even by many Social Democrats, the Anarchist deals
\newpage\noindent
%-----File: 016.png
his hardest blows at the sophisms that uphold the State, and
urges workers in striving for their emancipation to confine their
efforts to the economic field.

It follows, therefore, that politically and economically his
attitude is purely revolutionary; and hence arises the vilification
and misrepresentation that Anarchism, which denounces all
forms of social injustice, meets with in the press and from
public speakers.

Rightly conceived, Anarchism is no mere abstract ideal
theory of human society. It views life and social relations
with eyes disillusioned. Making an end of all superstitions,
prejudices, and false sentiments, it tries to see things as they
really are; and without building castles in the air, it finds by
the simple correlation of established facts that the grandest
possibilities of a full and free life can be placed within the
reach of all, once that monstrous bulwark of all our social
iniquities---the State---has been destroyed, and common property
declared.

By education, by free organisation, by individual and associated
resistance to political and economic tyranny, the Anarchist
hopes to achieve his aim. The task may seem impossible to
many, but it should be remembered that in science, in literature,
in art, the highest minds are with the Anarchists or are imbued
with distinct Anarchist tendencies. Even our bitterest opponents
admit the beauty of our ``dream,'' and reluctantly confess that it
would be well for humanity if it were ``possible.'' Anarchist
Communist propaganda is the intelligent, organised, determined
effort to realise the ``dream,'' and to ensure that freedom and
well-being for all \textit{shall} be possible.



\newpage\pagestyle{empty}
%-----File: 017.png




\begin{center}{\LARGE Modern Science and Anarchism.}\end{center}
\begin{center}{By \textsc{Peter Kropotkin}.}\end{center}


\begin{center}{A New and Revised Translation, with three additional chapters,
and a useful and interesting Glossary.}\end{center}

\begin{center}{112 pages; Paper Covers, 6d.\ net; also in Art Cambric, 1s.\ 6d.\ net.
Postage, paper 1$\frac{1}{2}$d., cloth 3d.}\end{center}

\begin{quote}
``As a survey of modern science in relation to society \dots this book would
be hard to beat.\dots The glossary of about 16 crowded pages is alone worth the
price of the volume.''---\textit{Maoriland Worker.}
\end{quote}

\DoubleLine


\begin{center}{\LARGE The Conquest of Bread.}\end{center}
\begin{center}{By \textsc{Peter Kropotkin}.}\end{center}

\begin{center}{A New and Cheaper Edition.  Cloth, 1s.\ net; postage 2$\frac{1}{2}$d.}\end{center}

\DoubleLine


\begin{center}{\LARGE God and the State.}\end{center}
\begin{center}{By \textsc{Michael Bakunin}.}\end{center}

\begin{center}{\textit{A new edition, revised from the original Manuscript.}\\
With a new Portrait.}\end{center}

\begin{center}{Paper cover, 6d.\ net; cloth, 1s.\ net. Postage 1d.\ and 2d.}\end{center}

\DoubleLine


\begin{center}{\large LIBERTY AND THE GREAT LIBERTARIANS.}\end{center}
\begin{center}{An Anthology on Liberty.}\end{center}

\begin{center}{Edited and Compiled, with Preface, Introduction, and Index, by\\
\textsc{Charles T. Sprading}.}\end{center}

\begin{quote}
Presenting quickly and succinctly the best utterances of the greatest
thinkers on every phase of human freedom. Many valuable quotations from
suppressed, ignored, and hitherto inaccessible sources.
\end{quote}

\begin{center}{Price 6s.\ 6d.\ net, postage 4d.}\end{center}

\DoubleLine

\begin{center}{\textsc{Freedom Press, 127 Ossulston Street, London, N.W.}}\end{center}



\newpage
%-----File: 018.png

\textit{Transcriber's note: on the image used for this edition, this page
was partly obscured by binding tape. Unrealisable sections are marked \dots}

\begin{center}{\large FREEDOM.}\end{center}

\begin{center}{\textsc{A Journal of Anarchist Communism.}}\end{center}

\begin{center}{(\textit{Established 1886.})}\end{center}

\begin{center}{\textit{Monthly, 1d.\ Annual Subscription, 1s.\ 6d.\ post free}}\end{center}

\medskip\hrule\medskip


\begin{center}{\large PAMPHLET AND BOOK LIST.}\end{center}

\noindent ANARCHIST COMMUNISM\@. By \textsc{Peter Kropotkin}. 1d.\\
ANARCHISM\@. By \textsc{Peter Kropotkin}. 1d.\\
ANARCHIST MORALITY\@. By \textsc{Peter Kropotkin}. 1d.\\
THE WAGE SYSTEM\@. By \textsc{Peter Kropotkin}. 1d.\\
A TALK ABOUT ANARCHIST COMMUNISM BETWEEN\\
\indent \textsc{Two Workers}. By \textsc{E. Malatesta}. 1d.\\
THE STATE: \textsc{Its Historic Role.} By \textsc{Peter Kropotkin}. \dots\\
EXPROPRIATION\@. By \textsc{Peter Kropotkin}. 1d.\\
LAW AND AUTHORITY\@. By \textsc{Peter Kropotkin}. 2d.\\
THE PYRAMID OF TYRANNY\@. By \textsc{D. Nieuwenhuis}. \dots\\
THE PLACE OF ANARCHISM IN SOCIALISTIC EVOLUTION.\\
\indent By \textsc{Peter Kropotkin}. 1d.\\
AN APPEAL TO THE YOUNG\@. By \textsc{Peter Kropotkin}. \dots\\
THE COMMUNE OF PARIS\@. By \textsc{Peter Kropotkin}. \dots\\
EVOLUTION AND REVOLUTION\@. BY \textsc{Elis\'{e}e Reclus}. \dots\\
THE SOCIAL GENERAL STRIKE\@. By \textsc{A. Roller}. \dots\\
THE CHICAGO MARTYRS\@. With Portraits. 1d.\\
DIRECT ACTION \textit{v.} LEGISLATION\@. By \textsc{J. Blair Smith}. \dots\\
WARS AND CAPITALISM\@. By \textsc{P. Kropotkin}. 1d

\SingleLine

\noindent THE GREAT FRENCH REVOLUTION, 1789-1793. By \textsc{Peter Kro\-pot\-kin}.\\
\indent 6s.\ net; postage 4d.\\
MUTUAL AID\@. By \textsc{Peter Kropotkin}. 3s.\ 6d.\ postage \dots\\
FIELDS, FACTORIES, AND WORKSHOPS\@. By \textsc{Peter Kro\-pot\-kin}. Cloth,\\
\indent 1s.\ net, postage 3d.\\
ANARCHISM\@. By Dr.\ \textsc{Paul Eltzbacher}. With \dots
6s.\ 6d.\ net, postage 4d.\\
NEWS FROM NOWHERE\@. By \textsc{Wm.\ Morris}. Cloth \dots
paper 1s.; postage 2d.\\
FAMOUS SPEECHES OF THE EIGHT CHICAGO ANARCH\-ISTS. 1s.\ 3d.,\\
\indent postage 2d.

\SingleLine

Orders, with cash (postage $\tfrac{1}{2}$d.\ each pamphlet), to
\textsc{Freedom Press, 127 Ossulston Street, London, N.W.}


\newpage
\small
\pagenumbering{Roman}
\begin{verbatim}
End of Project Gutenberg's The Right To Ignore The State, by Herbert Spencer

*** END OF THIS PROJECT GUTENBERG EBOOK THE RIGHT TO IGNORE THE STATE ***

***** This file should be named 34649-pdf.pdf or 34649-pdf.zip *****
This and all associated files of various formats will be found in:
        http://www.gutenberg.org/3/4/6/4/34649/

Produced by Fritz Ohrenschall, Keith Edkins and the Online
Distributed Proofreading Team at http://www.pgdp.net (This
file was produced from images generously made available
by The Internet Archive)


Updated editions will replace the previous one--the old editions
will be renamed.

Creating the works from public domain print editions means that no
one owns a United States copyright in these works, so the Foundation
(and you!) can copy and distribute it in the United States without
permission and without paying copyright royalties.  Special rules,
set forth in the General Terms of Use part of this license, apply to
copying and distributing Project Gutenberg-tm electronic works to
protect the PROJECT GUTENBERG-tm concept and trademark.  Project
Gutenberg is a registered trademark, and may not be used if you
charge for the eBooks, unless you receive specific permission.  If you
do not charge anything for copies of this eBook, complying with the
rules is very easy.  You may use this eBook for nearly any purpose
such as creation of derivative works, reports, performances and
research.  They may be modified and printed and given away--you may do
practically ANYTHING with public domain eBooks.  Redistribution is
subject to the trademark license, especially commercial
redistribution.



*** START: FULL LICENSE ***

THE FULL PROJECT GUTENBERG LICENSE
PLEASE READ THIS BEFORE YOU DISTRIBUTE OR USE THIS WORK

To protect the Project Gutenberg-tm mission of promoting the free
distribution of electronic works, by using or distributing this work
(or any other work associated in any way with the phrase "Project
Gutenberg"), you agree to comply with all the terms of the Full Project
Gutenberg-tm License (available with this file or online at
http://gutenberg.org/license).


Section 1.  General Terms of Use and Redistributing Project Gutenberg-tm
electronic works

1.A.  By reading or using any part of this Project Gutenberg-tm
electronic work, you indicate that you have read, understand, agree to
and accept all the terms of this license and intellectual property
(trademark/copyright) agreement.  If you do not agree to abide by all
the terms of this agreement, you must cease using and return or destroy
all copies of Project Gutenberg-tm electronic works in your possession.
If you paid a fee for obtaining a copy of or access to a Project
Gutenberg-tm electronic work and you do not agree to be bound by the
terms of this agreement, you may obtain a refund from the person or
entity to whom you paid the fee as set forth in paragraph 1.E.8.

1.B.  "Project Gutenberg" is a registered trademark.  It may only be
used on or associated in any way with an electronic work by people who
agree to be bound by the terms of this agreement.  There are a few
things that you can do with most Project Gutenberg-tm electronic works
even without complying with the full terms of this agreement.  See
paragraph 1.C below.  There are a lot of things you can do with Project
Gutenberg-tm electronic works if you follow the terms of this agreement
and help preserve free future access to Project Gutenberg-tm electronic
works.  See paragraph 1.E below.

1.C.  The Project Gutenberg Literary Archive Foundation ("the Foundation"
or PGLAF), owns a compilation copyright in the collection of Project
Gutenberg-tm electronic works.  Nearly all the individual works in the
collection are in the public domain in the United States.  If an
individual work is in the public domain in the United States and you are
located in the United States, we do not claim a right to prevent you from
copying, distributing, performing, displaying or creating derivative
works based on the work as long as all references to Project Gutenberg
are removed.  Of course, we hope that you will support the Project
Gutenberg-tm mission of promoting free access to electronic works by
freely sharing Project Gutenberg-tm works in compliance with the terms of
this agreement for keeping the Project Gutenberg-tm name associated with
the work.  You can easily comply with the terms of this agreement by
keeping this work in the same format with its attached full Project
Gutenberg-tm License when you share it without charge with others.

1.D.  The copyright laws of the place where you are located also govern
what you can do with this work.  Copyright laws in most countries are in
a constant state of change.  If you are outside the United States, check
the laws of your country in addition to the terms of this agreement
before downloading, copying, displaying, performing, distributing or
creating derivative works based on this work or any other Project
Gutenberg-tm work.  The Foundation makes no representations concerning
the copyright status of any work in any country outside the United
States.

1.E.  Unless you have removed all references to Project Gutenberg:

1.E.1.  The following sentence, with active links to, or other immediate
access to, the full Project Gutenberg-tm License must appear prominently
whenever any copy of a Project Gutenberg-tm work (any work on which the
phrase "Project Gutenberg" appears, or with which the phrase "Project
Gutenberg" is associated) is accessed, displayed, performed, viewed,
copied or distributed:

This eBook is for the use of anyone anywhere at no cost and with
almost no restrictions whatsoever.  You may copy it, give it away or
re-use it under the terms of the Project Gutenberg License included
with this eBook or online at www.gutenberg.org

1.E.2.  If an individual Project Gutenberg-tm electronic work is derived
from the public domain (does not contain a notice indicating that it is
posted with permission of the copyright holder), the work can be copied
and distributed to anyone in the United States without paying any fees
or charges.  If you are redistributing or providing access to a work
with the phrase "Project Gutenberg" associated with or appearing on the
work, you must comply either with the requirements of paragraphs 1.E.1
through 1.E.7 or obtain permission for the use of the work and the
Project Gutenberg-tm trademark as set forth in paragraphs 1.E.8 or
1.E.9.

1.E.3.  If an individual Project Gutenberg-tm electronic work is posted
with the permission of the copyright holder, your use and distribution
must comply with both paragraphs 1.E.1 through 1.E.7 and any additional
terms imposed by the copyright holder.  Additional terms will be linked
to the Project Gutenberg-tm License for all works posted with the
permission of the copyright holder found at the beginning of this work.

1.E.4.  Do not unlink or detach or remove the full Project Gutenberg-tm
License terms from this work, or any files containing a part of this
work or any other work associated with Project Gutenberg-tm.

1.E.5.  Do not copy, display, perform, distribute or redistribute this
electronic work, or any part of this electronic work, without
prominently displaying the sentence set forth in paragraph 1.E.1 with
active links or immediate access to the full terms of the Project
Gutenberg-tm License.

1.E.6.  You may convert to and distribute this work in any binary,
compressed, marked up, nonproprietary or proprietary form, including any
word processing or hypertext form.  However, if you provide access to or
distribute copies of a Project Gutenberg-tm work in a format other than
"Plain Vanilla ASCII" or other format used in the official version
posted on the official Project Gutenberg-tm web site (www.gutenberg.org),
you must, at no additional cost, fee or expense to the user, provide a
copy, a means of exporting a copy, or a means of obtaining a copy upon
request, of the work in its original "Plain Vanilla ASCII" or other
form.  Any alternate format must include the full Project Gutenberg-tm
License as specified in paragraph 1.E.1.

1.E.7.  Do not charge a fee for access to, viewing, displaying,
performing, copying or distributing any Project Gutenberg-tm works
unless you comply with paragraph 1.E.8 or 1.E.9.

1.E.8.  You may charge a reasonable fee for copies of or providing
access to or distributing Project Gutenberg-tm electronic works provided
that

- You pay a royalty fee of 20% of the gross profits you derive from
     the use of Project Gutenberg-tm works calculated using the method
     you already use to calculate your applicable taxes.  The fee is
     owed to the owner of the Project Gutenberg-tm trademark, but he
     has agreed to donate royalties under this paragraph to the
     Project Gutenberg Literary Archive Foundation.  Royalty payments
     must be paid within 60 days following each date on which you
     prepare (or are legally required to prepare) your periodic tax
     returns.  Royalty payments should be clearly marked as such and
     sent to the Project Gutenberg Literary Archive Foundation at the
     address specified in Section 4, "Information about donations to
     the Project Gutenberg Literary Archive Foundation."

- You provide a full refund of any money paid by a user who notifies
     you in writing (or by e-mail) within 30 days of receipt that s/he
     does not agree to the terms of the full Project Gutenberg-tm
     License.  You must require such a user to return or
     destroy all copies of the works possessed in a physical medium
     and discontinue all use of and all access to other copies of
     Project Gutenberg-tm works.

- You provide, in accordance with paragraph 1.F.3, a full refund of any
     money paid for a work or a replacement copy, if a defect in the
     electronic work is discovered and reported to you within 90 days
     of receipt of the work.

- You comply with all other terms of this agreement for free
     distribution of Project Gutenberg-tm works.

1.E.9.  If you wish to charge a fee or distribute a Project Gutenberg-tm
electronic work or group of works on different terms than are set
forth in this agreement, you must obtain permission in writing from
both the Project Gutenberg Literary Archive Foundation and Michael
Hart, the owner of the Project Gutenberg-tm trademark.  Contact the
Foundation as set forth in Section 3 below.

1.F.

1.F.1.  Project Gutenberg volunteers and employees expend considerable
effort to identify, do copyright research on, transcribe and proofread
public domain works in creating the Project Gutenberg-tm
collection.  Despite these efforts, Project Gutenberg-tm electronic
works, and the medium on which they may be stored, may contain
"Defects," such as, but not limited to, incomplete, inaccurate or
corrupt data, transcription errors, a copyright or other intellectual
property infringement, a defective or damaged disk or other medium, a
computer virus, or computer codes that damage or cannot be read by
your equipment.

1.F.2.  LIMITED WARRANTY, DISCLAIMER OF DAMAGES - Except for the "Right
of Replacement or Refund" described in paragraph 1.F.3, the Project
Gutenberg Literary Archive Foundation, the owner of the Project
Gutenberg-tm trademark, and any other party distributing a Project
Gutenberg-tm electronic work under this agreement, disclaim all
liability to you for damages, costs and expenses, including legal
fees.  YOU AGREE THAT YOU HAVE NO REMEDIES FOR NEGLIGENCE, STRICT
LIABILITY, BREACH OF WARRANTY OR BREACH OF CONTRACT EXCEPT THOSE
PROVIDED IN PARAGRAPH 1.F.3.  YOU AGREE THAT THE FOUNDATION, THE
TRADEMARK OWNER, AND ANY DISTRIBUTOR UNDER THIS AGREEMENT WILL NOT BE
LIABLE TO YOU FOR ACTUAL, DIRECT, INDIRECT, CONSEQUENTIAL, PUNITIVE OR
INCIDENTAL DAMAGES EVEN IF YOU GIVE NOTICE OF THE POSSIBILITY OF SUCH
DAMAGE.

1.F.3.  LIMITED RIGHT OF REPLACEMENT OR REFUND - If you discover a
defect in this electronic work within 90 days of receiving it, you can
receive a refund of the money (if any) you paid for it by sending a
written explanation to the person you received the work from.  If you
received the work on a physical medium, you must return the medium with
your written explanation.  The person or entity that provided you with
the defective work may elect to provide a replacement copy in lieu of a
refund.  If you received the work electronically, the person or entity
providing it to you may choose to give you a second opportunity to
receive the work electronically in lieu of a refund.  If the second copy
is also defective, you may demand a refund in writing without further
opportunities to fix the problem.

1.F.4.  Except for the limited right of replacement or refund set forth
in paragraph 1.F.3, this work is provided to you 'AS-IS' WITH NO OTHER
WARRANTIES OF ANY KIND, EXPRESS OR IMPLIED, INCLUDING BUT NOT LIMITED TO
WARRANTIES OF MERCHANTIBILITY OR FITNESS FOR ANY PURPOSE.

1.F.5.  Some states do not allow disclaimers of certain implied
warranties or the exclusion or limitation of certain types of damages.
If any disclaimer or limitation set forth in this agreement violates the
law of the state applicable to this agreement, the agreement shall be
interpreted to make the maximum disclaimer or limitation permitted by
the applicable state law.  The invalidity or unenforceability of any
provision of this agreement shall not void the remaining provisions.

1.F.6.  INDEMNITY - You agree to indemnify and hold the Foundation, the
trademark owner, any agent or employee of the Foundation, anyone
providing copies of Project Gutenberg-tm electronic works in accordance
with this agreement, and any volunteers associated with the production,
promotion and distribution of Project Gutenberg-tm electronic works,
harmless from all liability, costs and expenses, including legal fees,
that arise directly or indirectly from any of the following which you do
or cause to occur: (a) distribution of this or any Project Gutenberg-tm
work, (b) alteration, modification, or additions or deletions to any
Project Gutenberg-tm work, and (c) any Defect you cause.


Section  2.  Information about the Mission of Project Gutenberg-tm

Project Gutenberg-tm is synonymous with the free distribution of
electronic works in formats readable by the widest variety of computers
including obsolete, old, middle-aged and new computers.  It exists
because of the efforts of hundreds of volunteers and donations from
people in all walks of life.

Volunteers and financial support to provide volunteers with the
assistance they need, are critical to reaching Project Gutenberg-tm's
goals and ensuring that the Project Gutenberg-tm collection will
remain freely available for generations to come.  In 2001, the Project
Gutenberg Literary Archive Foundation was created to provide a secure
and permanent future for Project Gutenberg-tm and future generations.
To learn more about the Project Gutenberg Literary Archive Foundation
and how your efforts and donations can help, see Sections 3 and 4
and the Foundation web page at http://www.pglaf.org.


Section 3.  Information about the Project Gutenberg Literary Archive
Foundation

The Project Gutenberg Literary Archive Foundation is a non profit
501(c)(3) educational corporation organized under the laws of the
state of Mississippi and granted tax exempt status by the Internal
Revenue Service.  The Foundation's EIN or federal tax identification
number is 64-6221541.  Its 501(c)(3) letter is posted at
http://pglaf.org/fundraising.  Contributions to the Project Gutenberg
Literary Archive Foundation are tax deductible to the full extent
permitted by U.S. federal laws and your state's laws.

The Foundation's principal office is located at 4557 Melan Dr. S.
Fairbanks, AK, 99712., but its volunteers and employees are scattered
throughout numerous locations.  Its business office is located at
809 North 1500 West, Salt Lake City, UT 84116, (801) 596-1887, email
business@pglaf.org.  Email contact links and up to date contact
information can be found at the Foundation's web site and official
page at http://pglaf.org

For additional contact information:
     Dr. Gregory B. Newby
     Chief Executive and Director
     gbnewby@pglaf.org


Section 4.  Information about Donations to the Project Gutenberg
Literary Archive Foundation

Project Gutenberg-tm depends upon and cannot survive without wide
spread public support and donations to carry out its mission of
increasing the number of public domain and licensed works that can be
freely distributed in machine readable form accessible by the widest
array of equipment including outdated equipment.  Many small donations
($1 to $5,000) are particularly important to maintaining tax exempt
status with the IRS.

The Foundation is committed to complying with the laws regulating
charities and charitable donations in all 50 states of the United
States.  Compliance requirements are not uniform and it takes a
considerable effort, much paperwork and many fees to meet and keep up
with these requirements.  We do not solicit donations in locations
where we have not received written confirmation of compliance.  To
SEND DONATIONS or determine the status of compliance for any
particular state visit http://pglaf.org

While we cannot and do not solicit contributions from states where we
have not met the solicitation requirements, we know of no prohibition
against accepting unsolicited donations from donors in such states who
approach us with offers to donate.

International donations are gratefully accepted, but we cannot make
any statements concerning tax treatment of donations received from
outside the United States.  U.S. laws alone swamp our small staff.

Please check the Project Gutenberg Web pages for current donation
methods and addresses.  Donations are accepted in a number of other
ways including checks, online payments and credit card donations.
To donate, please visit: http://pglaf.org/donate


Section 5.  General Information About Project Gutenberg-tm electronic
works.

Professor Michael S. Hart is the originator of the Project Gutenberg-tm
concept of a library of electronic works that could be freely shared
with anyone.  For thirty years, he produced and distributed Project
Gutenberg-tm eBooks with only a loose network of volunteer support.


Project Gutenberg-tm eBooks are often created from several printed
editions, all of which are confirmed as Public Domain in the U.S.
unless a copyright notice is included.  Thus, we do not necessarily
keep eBooks in compliance with any particular paper edition.


Most people start at our Web site which has the main PG search facility:

     http://www.gutenberg.org

This Web site includes information about Project Gutenberg-tm,
including how to make donations to the Project Gutenberg Literary
Archive Foundation, how to help produce our new eBooks, and how to
subscribe to our email newsletter to hear about new eBooks.

\end{verbatim}
% %%%%%%%%%%%%%%%%%%%%%%%%%%%%%%%%%%%%%%%%%%%%%%%%%%%%%%%%%%%%%%%%%%%%%%% %
%                                                                         %
% End of Project Gutenberg's The Right To Ignore The State, by Herbert Spencer
%                                                                         %
% *** END OF THIS PROJECT GUTENBERG EBOOK THE RIGHT TO IGNORE THE STATE ***
%                                                                         %
% ***** This file should be named 34649-t.tex or 34649-t.zip *****        %
% This and all associated files of various formats will be found in:      %
%         http://www.gutenberg.org/3/4/6/4/34649/                         %
%                                                                         %
% %%%%%%%%%%%%%%%%%%%%%%%%%%%%%%%%%%%%%%%%%%%%%%%%%%%%%%%%%%%%%%%%%%%%%%% %

\end{document}

### lprep configuration
@ControlwordReplace = (
  ['\Definition',"Definition"],
  ['\S',"Section"],
  ['\pounds',"L"]
  );

###
This is pdfTeXk, Version 3.141592-1.40.3 (Web2C 7.5.6) (format=pdflatex 2010.5.6)  14 DEC 2010 01:38
entering extended mode
 %&-line parsing enabled.
**34649-t.tex
(./34649-t.tex
LaTeX2e <2005/12/01>
Babel <v3.8h> and hyphenation patterns for english, usenglishmax, dumylang, noh
yphenation, arabic, farsi, croatian, ukrainian, russian, bulgarian, czech, slov
ak, danish, dutch, finnish, basque, french, german, ngerman, ibycus, greek, mon
ogreek, ancientgreek, hungarian, italian, latin, mongolian, norsk, icelandic, i
nterlingua, turkish, coptic, romanian, welsh, serbian, slovenian, estonian, esp
eranto, uppersorbian, indonesian, polish, portuguese, spanish, catalan, galicia
n, swedish, ukenglish, pinyin, loaded.
(/usr/share/texmf-texlive/tex/latex/base/book.cls
Document Class: book 2005/09/16 v1.4f Standard LaTeX document class
(/usr/share/texmf-texlive/tex/latex/base/bk10.clo
File: bk10.clo 2005/09/16 v1.4f Standard LaTeX file (size option)
)
\c@part=\count79
\c@chapter=\count80
\c@section=\count81
\c@subsection=\count82
\c@subsubsection=\count83
\c@paragraph=\count84
\c@subparagraph=\count85
\c@figure=\count86
\c@table=\count87
\abovecaptionskip=\skip41
\belowcaptionskip=\skip42
\bibindent=\dimen102
) (/usr/share/texmf-texlive/tex/latex/base/inputenc.sty
Package: inputenc 2006/05/05 v1.1b Input encoding file
\inpenc@prehook=\toks14
\inpenc@posthook=\toks15
(/usr/share/texmf-texlive/tex/latex/base/latin1.def
File: latin1.def 2006/05/05 v1.1b Input encoding file
)) (/usr/share/texmf-texlive/tex/latex/amsmath/amsmath.sty
Package: amsmath 2000/07/18 v2.13 AMS math features
\@mathmargin=\skip43
For additional information on amsmath, use the `?' option.
(/usr/share/texmf-texlive/tex/latex/amsmath/amstext.sty
Package: amstext 2000/06/29 v2.01
(/usr/share/texmf-texlive/tex/latex/amsmath/amsgen.sty
File: amsgen.sty 1999/11/30 v2.0
\@emptytoks=\toks16
\ex@=\dimen103
)) (/usr/share/texmf-texlive/tex/latex/amsmath/amsbsy.sty
Package: amsbsy 1999/11/29 v1.2d
\pmbraise@=\dimen104
) (/usr/share/texmf-texlive/tex/latex/amsmath/amsopn.sty
Package: amsopn 1999/12/14 v2.01 operator names
)
\inf@bad=\count88
LaTeX Info: Redefining \frac on input line 211.
\uproot@=\count89
\leftroot@=\count90
LaTeX Info: Redefining \overline on input line 307.
\classnum@=\count91
\DOTSCASE@=\count92
LaTeX Info: Redefining \ldots on input line 379.
LaTeX Info: Redefining \dots on input line 382.
LaTeX Info: Redefining \cdots on input line 467.
\Mathstrutbox@=\box26
\strutbox@=\box27
\big@size=\dimen105
LaTeX Font Info:    Redeclaring font encoding OML on input line 567.
LaTeX Font Info:    Redeclaring font encoding OMS on input line 568.
\macc@depth=\count93
\c@MaxMatrixCols=\count94
\dotsspace@=\muskip10
\c@parentequation=\count95
\dspbrk@lvl=\count96
\tag@help=\toks17
\row@=\count97
\column@=\count98
\maxfields@=\count99
\andhelp@=\toks18
\eqnshift@=\dimen106
\alignsep@=\dimen107
\tagshift@=\dimen108
\tagwidth@=\dimen109
\totwidth@=\dimen110
\lineht@=\dimen111
\@envbody=\toks19
\multlinegap=\skip44
\multlinetaggap=\skip45
\mathdisplay@stack=\toks20
LaTeX Info: Redefining \[ on input line 2666.
LaTeX Info: Redefining \] on input line 2667.
) (/usr/share/texmf-texlive/tex/latex/tools/verbatim.sty
Package: verbatim 2003/08/22 v1.5q LaTeX2e package for verbatim enhancements
\every@verbatim=\toks21
\verbatim@line=\toks22
\verbatim@in@stream=\read1
)
No file 34649-t.aux.
\openout1 = `34649-t.aux'.

LaTeX Font Info:    Checking defaults for OML/cmm/m/it on input line 95.
LaTeX Font Info:    ... okay on input line 95.
LaTeX Font Info:    Checking defaults for T1/cmr/m/n on input line 95.
LaTeX Font Info:    ... okay on input line 95.
LaTeX Font Info:    Checking defaults for OT1/cmr/m/n on input line 95.
LaTeX Font Info:    ... okay on input line 95.
LaTeX Font Info:    Checking defaults for OMS/cmsy/m/n on input line 95.
LaTeX Font Info:    ... okay on input line 95.
LaTeX Font Info:    Checking defaults for OMX/cmex/m/n on input line 95.
LaTeX Font Info:    ... okay on input line 95.
LaTeX Font Info:    Checking defaults for U/cmr/m/n on input line 95.
LaTeX Font Info:    ... okay on input line 95.
[1

{/var/lib/texmf/fonts/map/pdftex/updmap/pdftex.map}] [2] [3]
LaTeX Font Info:    Try loading font information for OMS+cmr on input line 208.

(/usr/share/texmf-texlive/tex/latex/base/omscmr.fd
File: omscmr.fd 1999/05/25 v2.5h Standard LaTeX font definitions
)
LaTeX Font Info:    Font shape `OMS/cmr/m/n' in size <10> not available
(Font)              Font shape `OMS/cmsy/m/n' tried instead on input line 208.
[4] [5] [6] [7] [8] [9] [10] [11] [12] [13] [14] [15] [16] [17] [1] [2] [3] [4]
[5] [6] [7] [8] (./34649-t.aux)

 *File List*
    book.cls    2005/09/16 v1.4f Standard LaTeX document class
    bk10.clo    2005/09/16 v1.4f Standard LaTeX file (size option)
inputenc.sty    2006/05/05 v1.1b Input encoding file
  latin1.def    2006/05/05 v1.1b Input encoding file
 amsmath.sty    2000/07/18 v2.13 AMS math features
 amstext.sty    2000/06/29 v2.01
  amsgen.sty    1999/11/30 v2.0
  amsbsy.sty    1999/11/29 v1.2d
  amsopn.sty    1999/12/14 v2.01 operator names
verbatim.sty    2003/08/22 v1.5q LaTeX2e package for verbatim enhancements
  omscmr.fd    1999/05/25 v2.5h Standard LaTeX font definitions
 ***********

 ) 
Here is how much of TeX's memory you used:
 954 strings out of 94074
 11006 string characters out of 1165154
 67772 words of memory out of 1500000
 4282 multiletter control sequences out of 10000+50000
 10701 words of font info for 39 fonts, out of 1200000 for 2000
 645 hyphenation exceptions out of 8191
 26i,8n,23p,235b,286s stack positions out of 5000i,500n,6000p,200000b,5000s
</usr/share/texmf-texlive/fonts/type1/bluesky/cm/cmcsc10.pfb></usr/share/texm
f-texlive/fonts/type1/bluesky/cm/cmr10.pfb></usr/share/texmf-texlive/fonts/type
1/bluesky/cm/cmr12.pfb></usr/share/texmf-texlive/fonts/type1/bluesky/cm/cmr17.p
fb></usr/share/texmf-texlive/fonts/type1/bluesky/cm/cmr6.pfb></usr/share/texmf-
texlive/fonts/type1/bluesky/cm/cmr7.pfb></usr/share/texmf-texlive/fonts/type1/b
luesky/cm/cmr8.pfb></usr/share/texmf-texlive/fonts/type1/bluesky/cm/cmsy10.pfb>
</usr/share/texmf-texlive/fonts/type1/bluesky/cm/cmti10.pfb></usr/share/texmf-t
exlive/fonts/type1/bluesky/cm/cmti8.pfb></usr/share/texmf-texlive/fonts/type1/b
luesky/cm/cmtt9.pfb></usr/share/texmf-texlive/fonts/type1/bluesky/cm/cmu10.pfb>
Output written on 34649-t.pdf (25 pages, 137831 bytes).
PDF statistics:
 131 PDF objects out of 1000 (max. 8388607)
 0 named destinations out of 1000 (max. 131072)
 1 words of extra memory for PDF output out of 10000 (max. 10000000)

